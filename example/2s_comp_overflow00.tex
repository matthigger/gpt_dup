\begin{enumerate}[i]
    \item Give the 8-bit two's complement representations of
    the following decimal values: $55$, $83$, $-79$, $-88$.
    Give some justification with responses for any negative values.

    \answer{

        The positive values are the same as the unsigned binary:
        \begin{center}
            \begin{math}
                \begin{array}{ccccc}
                    55 & = & 32+16+4+2+1 & = & 00110111 \\
                    83 & = & 64+16+2+1   & = & 01010011 \\
                \end{array}
            \end{math}
        \end{center}

        To make the remaining values negative, we'll need to activate that leading bit which adds $-2^7$ to our value (see solution immediately above). So that the sum of all digits is -79 we must have $-128 + x = -79 \rightarrow x = 49$. That is, the 7 rightward bits must sum to 49. Written as an 8 bit unsigned value $49 = (00110001)_2$ so that $-79 = (10110001)_2$ in 2's complement.

        Similarly,
        \begin{equation*}
            -128 + x = -88 \rightarrow x = 40 = (00101000)_2 \rightarrow -88 = (10101000)_2 \text{ in 2's complement}
        \end{equation*}

    }

    \rubric{
        For each of the positive values (55, 83):
        \begin{itemize}
            \item 2 pts final correct answer (no work needs to be shown)
        \end{itemize}

        For each of the negative values (-79, -88):

        \begin{itemize}
            \item 3 pts final correct answer (-1 per bit wrong, no credit if the leading bit is not 1)
            \item 2 pts some justification shown (some computation as shown in solution or flip all bits \& add one is acceptable)
        \end{itemize}

    }

    \item Compute the following operations as a computer would, directly from the (8-bit) 2's complement representations: $-79 + 55$, $-79 - 88$, and $83 + 55$.
    Indicate whether each operation results in an overflow or not.

    \emph{Note:\/} Use the two's complement representations from part~ii above.

    \answer{
        \begin{center}
            \begin{math}
                \begin{array}{cccccccccccr}
                    & & 1 & 0^1 & 1^1 & 1 & 0^1 & 0^1 & 0^1 & 1 & = & -79 \\
                    + & & 0 & 0   & 1   & 1 & 0   & 1   & 1   & 1 & = & 55  \\
                    \hline
                    & & 1 & 1   & 1   & 0 & 1   & 0   & 0   & 0 & = & -24
                \end{array}
            \end{math}
            There is \textbf{no overflow}
        \end{center}


        \begin{center}
            \begin{math}
                \begin{array}{cccccccccccr}
                    & & 1 & 0^1 & 1 & 1 & 0 & 0 & 0 & 1 & =    & -79 \\
                    + & & 1 & 0   & 1 & 0 & 1 & 0 & 0 & 0 & =    & -88 \\
                    \hline
                    & & 0 & 1   & 0 & 1 & 1 & 0 & 0 & 1 & \neq & 89
                \end{array}
            \end{math}
            There is an \textbf{overflow}
        \end{center}

        \begin{center}
            \begin{math}
                \begin{array}{cccccccccccr}
                    & & 0^1 & 1^1 & 0^1 & 1 & 0^1 & 0^1 & 1^1 & 1 & =    & 83   \\
                    + & & 0   & 0   & 1   & 1 & 0   & 1   & 1   & 1 & =    & 55   \\
                    \hline
                    & & 1   & 0   & 0   & 0 & 1   & 0   & 1   & 0 & \neq & -118
                \end{array}
            \end{math}
            There is an \textbf{overflow}
        \end{center}

    }
    \rubric{
        For each problem:
        \begin{itemize}
            \item 1 pt addition is completely correct
            \item 2 pt overflow or not correctly given
        \end{itemize}
    }
\end{enumerate}
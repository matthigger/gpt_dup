Sure, let's rewrite the math problem with new numbers and update the solutions accordingly.

"\begin{enumerate}[i]
    \item Give the 8-bit two's complement representations of
    the following decimal values: $37$, $92$, $-47$, $-65$.
    Give some justification with responses for any negative values.

    \answer{

        The positive values are the same as the unsigned binary:
        \begin{center}
            \begin{math}
                \begin{array}{ccccc}
                    37 & = & 32+4+1 & = & 00100101 \\
                    92 & = & 64+16+8+4 & = & 01011100 \\
                \end{array}
            \end{math}
        \end{center}

        To make the remaining values negative, we'll need to activate that leading bit which adds $-2^7$ to our value. So, for the sum of all digits to be -47, we must have $-128 + x = -47 \rightarrow x = 81$. That is, the 7 rightward bits must sum to 81. Written as an 8-bit unsigned value $81 = (01010001)_2$ so that $-47 = (11010001)_2$ in 2's complement.

        Similarly,
        \begin{equation*}
            -128 + x = -65 \rightarrow x = 63 = (00111111)_2 \rightarrow -65 = (11000001)_2 \text{ in 2's complement}
        \end{equation*}

    }

    \rubric{
        For each of the positive values (37, 92):
        \begin{itemize}
            \item 2 pts final correct answer (no work needs to be shown)
        \end{itemize}

        For each of the negative values (-47, -65):

        \begin{itemize}
            \item 3 pts final correct answer (-1 per bit wrong, no credit if the leading bit is not 1)
            \item 2 pts some justification shown (some computation as shown in solution or flip all bits \& add one is acceptable)
        \end{itemize}

    }

    \item Compute the following operations as a computer would, directly from the (8-bit) 2's complement representations: $-47 + 37$, $-47 - 65$, and $92 + 37$.
    Indicate whether each operation results in an overflow or not.

    \emph{Note:\/} Use the two's complement representations from part~ii above.

    \answer{
        \begin{center}
            \begin{math}
                \begin{array}{cccccccccccr}
                    & & 1 & 1^1 & 0^1 & 1 & 0^1 & 0 & 0^1 & 1 & = & -47 \\
                    + & & 0 & 0   & 1   & 0 & 0   & 1   & 0   & 1 & = & 37  \\
                    \hline
                    & & 0 & 0   & 0   & 1 & 1   & 1   & 0   & 0 & = & -10
                \end{array}
            \end{math}
            There is \textbf{no overflow}
        \end{center}


        \begin{center}
            \begin{math}
                \begin{array}{cccccccccccr}
                    & & 1 & 1   & 0^1 & 1   & 0 & 0 & 0 & 1 & =    & -47 \\
                    + & & 1 & 1   & 0 & 0 & 0 & 0 & 0 & 1 & =    & -65 \\
                    \hline
                    & & 1 & 0   & 0   & 1 & 0 & 0 & 0 & 0 & \neq & -112
                \end{array}
            \end{math}
            There is an \textbf{overflow}
        \end{center}

        \begin{center}
            \begin{math}
                \begin{array}{cccccccccccr}
                    & & 0^1 & 1^1 & 0   & 1 & 1^1 & 1   & 0   & 0 & =    & 92   \\
                    + & & 0   & 0   & 1   & 0 & 0   & 1   & 0   & 1 & =    & 37   \\
                    \hline
                    & & 1   & 1   & 0   & 0 & 0   & 0   & 1   & 1 & \neq & -121
                \end{array}
            \end{math}
            There is an \textbf{overflow}
        \end{center}

    }
    \rubric{
        For each problem:
        \begin{itemize}
            \item 1 pt addition is completely correct
            \item 2 pt overflow or not correctly given
        \end{itemize}
    }
\end{enumerate}"